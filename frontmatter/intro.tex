\setcounter{chapter}{-1} 
\chapter{Dissertation Introduction} 
The operations literature has a rich history of applying formal mathematical models to study product and service settings. But since systems are composed of people, these systems also “involve human judgement and decision-making” \citep[p. 325]{Allon2019}. Modeling this judgement requires behavioral assumptions, such as “People are deterministic and predictable” \citep[p. 183]{Boudreau2003}. But reality is more nuanced: Quite often, the agents in our systems have discretion. My research empirically uses experimental and archival methods to study the impact of human discretion on a range of operational outcomes. Discretion shapes decisions. Decisions shape outcomes. And these outcomes coalesce to ultimately determine firm-level output – for better or worse. Indeed, one study warns that “... discretion has costs that need to be balanced against its potential benefits” settings. But since systems are composed of people, these systems also “involve human judgement and decision-making” \citep[p. 4404]{Ibanez2019}. By characterizing these costs and benefits, my research improves our field’s ability to generate models which value both descriptive power and analytical tractability. 

As a healthcare researcher, I primarily focus on the operational effects of discretion in the healthcare setting, where the literature frequently examines how providers shape a service system. Even so, my broader interest in discretion relates to the general nature of work, and my findings naturally generalize to other contexts. Across all my research, I maintain that operations management considers many facets of work: “What work should we do? When should we do it? How should we do it? And who should be doing it?” My research empirically responds to each of these questions with modern econometric and machine learning methods. Given my focus on the role of people within the system, my work provides valuable clarity into how human discretion affects operational outcomes, and my insights empower future operations research to better understand the full spectrum of worker behavior.

In the first chapter of my dissertation (also my job market paper), co-authored with my advisor Dr. Bradley Staats, entitled “Focusing Provider Attention: An Empirical Examination of Incentives and Feedback in Flu Vaccinations,” I blend work on operational compliance and healthcare operations to investigate how to improve healthcare clinic flu vaccination rates by altering provider behavior. To this end, I implement a field experiment to study a flu vaccine intervention among 145 clinics from 9 different states. This intervention randomly assigned these clinics to a control group or one of two separate treatment groups that received either relative performance feedback or financial incentives. 

Overall, I find that clinic-level performance feedback can effectively drive operational improvement, as relative performance feedback led to an 9.8 percentage point increase in flu vaccines. Perhaps more surprisingly, I find that firms receiving relative performance feedback significantly outperform incented firms. Few other studies compare the impact of these treatments at the firm-level, and even these exceptions cannot identify which is more effective. This study also highlights the important implications for public policy, as even a 1 percentage point increase in the U.S. adult flu vaccination rate could confer nearly \$400 million in societal benefits. The results of this forthcoming study add to the field of people-centric operations whereby workers productively use their discretion to improve operational performance.

In the second chapter of my dissertation, co-authored with Dr. Bradley Staats and Dr. Diwas KC, “Task Selection and Patient ‘Pick-up’ – How Familiarity Encourages Physician Multitasking in the Emergency Department,” I consider how familiarity between peer physicians affects patient selection and the chosen multitasking level, a process more commonly known in the Emergency Department (ED) as “patient pick-up." Using empirical observations from two Emergency Departments, we explore whether familiarity alters patient pick-up behavior, we determine the effect of familiarity on multitasking, and we measure the combined impact of familiarity and multitasking on other ED outcomes.

Overall, within more familiar groups, physicians appear willing to exert more effort. Greater average familiarity leads to an increase in patient pick-up rate, observed multitasking, and shorter patient wait time -- with no identifiable, negative impact to patient processing time or length of stay. Moreover, the effects intensify at the end of a physician’s shift and for patients in severe condition. By integrating two streams of literature, I reconcile two disparate approaches: (i) the familiarity literature ignores server discretion in task selection, and (ii) the modeling literature assumes servers in a group act individually regardless of group familiarity. These findings illustrate how disregarding behavioral facets of discretion might induce greater gaps between model predictions and observational outcomes.
	
In the final chapter of my dissertation, co-authored with Dr. Bradley Staats, entitled “The Value Behind Traffic: How Patient Discretion and Information Coarseness Altered Healthcare Traffic in the Wake of COVID-1,” I shift and consider discretion from the perspective of a patient or consumer. Traffic represents a key operational input, especially in healthcare where many aspects of medicine cannot happen from a distance. And yet with the emergence of Covid-19 in March of 2020, traffic to many clinics vanished overnight. I explore how the value of the service consumed impacts healthcare clinic traffic, and I also examine how traffic changed with coarse and precise signals of the pandemic environment.

To this end, my study combines observations from healthcare clinics across 15 US states, anonymized cellphone mobility data, Covid-19 severity measures, and state-wide stay-at-home orders. A Lasso-based procedure from machine learning non-arbitrarily selects instruments and generates exogenous county-level measures of individual mobility. I find clinic traffic returns with rising personal mobility, but the size of returns vary by the type of service: A 1\% increase in mobility yields a 2\% return to vaccination traffic but only a 0.5\% return to non-vaccination traffic. And the response to signals depends on signal severity and treatment sought: While precise signals affect non-vaccination traffic more than coarse signals, vaccination traffic seems immune to such factors. Contrary to the fears of many, I find patients prioritized important services and shouldered some of the burden of their own wellness. By studying such decisions, I identify and characterize powerful predictors of healthcare clinic traffic, and I encourage researchers to further explore the role of patient discretion in the co-production of wellness.
