
% 2″ margin at the top of the page; the second page, if any, returns to a 1″ top margin
% The heading “ABSTRACT” centered in all capital letters at top of page
% Your full name followed by title exactly as is on title page, centered and 1 double-spaced line below “ABSTRACT”
% The phrase “(Under the direction of [advisor's name])” in parentheses, centered, and 1 single-spaced line below title
% The text of your abstract must be double-spaced and no longer than 350 words for a dissertation
% Pages are numbered in lower case Roman numerals

\begin{center}
\vspace*{52pt}                  % RJN: Had to fine tune this, ended up setting same as "acknowledgements" page
{\normalfont\textbf{ABSTRACT}}
\vspace{11pt}

\begin{singlespace}
Robert Jay Niewoehner III: The Human Factor: The Behavioral Drivers and Operational Impact of Discretion \\
(Under the direction of Bradley Staats)
\end{singlespace}
\end{center}


\noindent
\textbf{Aim:} Recent operations research acknowledges that agents in our operational systems have discretion to make decisions. Modeling this behavior requires assumptions, but these assumptions may induce gaps between models and real-world observations. In the end, these decisions coalesce firm-level outputs, both for good and ill. Despite this, deliberate system design can transform problematic deviance into productive discretion. In this dissertation, I detail three explorations of system design and the operational impact of human discretion. \\
\textbf{Background:} The operations literature has a rich history of applying formal mathematical models to explain and study both product and service settings. Operational systems matter, but wherever these systems contain human discretion, people matter too. \\
\textbf{Context and Methodology:} I primarily focus on the operational effects of discretion in the healthcare setting, where the literature frequently examines how providers shape a service system. My research empirically responds to each of my research questions with modern econometric and machine learning methods. In the first essay, I designed and implemented a field experiment among 145 healthcare clinics. In the second and third essays, I leverage archival data analysis methods. \\
\textbf{Conclusion:} Operations research considers many facets of work: “What work should we do? When should we do it? How should we do it? And who should be doing it?” Given my focus on the role of people within the system, my work provides valuable clarity into how human discretion affects operational outcomes, and my insights empower future operations research to better understand the full spectrum of worker behavior. \\


\clearpage
